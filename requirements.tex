\documentclass[a4paper]{article}


\title{Robocup Junior Registration}
\author{Monika Buchalová, Filip Kotoč, Oliver Sidor, Filip Sršeň}
\usepackage{hyperref}

\begin{document}
	\maketitle
	
	\tableofcontents
	\newpage
	\section{Introduction}
	\label{sec:Intro}
	\subsection{Purpose of requirements document}
	\label{sub:purpose}
	\subsection{Scope of the product}
	\label{sub:scope}
	\subsection{Definitions, acronyms and abbreviations}
	\label{sub:definitios}
	\subsection{References}
	\label{sub:references}
	\subsection{Overview of the remainder of the document}
	\label{sub:overview}
	
	\newpage
	
	\section{General description}
	\label{sec:general-desc}
	\subsection{Product perspective}
	\label{sub:perspective}
	\subsection{Product functions}
	\label{sub:functions}
	\subsubsection{Team leader registration}
	User vyplni formular obsahujuci formular, kde vyplni meno, email. Potom dostane na mailom unikatny link/kod registracie, pod ktorym bude vediet upravovat svoju registraciu.
	\subsubsection{Individual registration}
	Team leader can add individuals under their registration. Then are prompted to fill out a form with the personal details.
	\subsubsection{Team assembly}
	Team leader can create and assemble teams out of registered individuals. They can register team into categories.
	\subsubsection{Organizer registration}
	Organizer fills out a default registration form with password and gets send email to confirm their email address 
	\subsection{User characteristics}
	\label{sub:users}
	\subsubsection{Team leader}
	Team leader is user who registers a team or multiple teams. Core functionalities for team leaders are:
	\begin{itemize}
		\item Registering individual competitors.
		\item Assembling teams from registered individuals.
		\item  
	\end{itemize} 
	\subsection{General constraints}
	\label{sub:constraints}
	\subsection{Assumptions and dependencies}
	\label{sub:dependencies}
	
	\newpage
	
	\section{Specific requirements}
	\label{sec:specific}
\end{document}

